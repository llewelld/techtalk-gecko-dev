% Imports
\usepackage{listings}
\usepackage{amsfonts}
\usepackage{amsmath}
\usepackage{pythonhighlight}
\usepackage{pgfpages}
\usepackage{tabularx}
\usepackage{ifxetex,ifluatex}
\usepackage{etoolbox}
\usepackage{framed}

\mode<handout>{%
	\pgfpagesuselayout{8 on 1}[a4paper,border shrink=5mm]
	\setbeameroption{show notes}
}

% Graphics configuration
\graphicspath{{./graphics/}}
\DeclareGraphicsExtensions{.pdf,.jpeg,.png,.jpg,.pdf}

% Useful macros
\def\etal{{\it et al.}}
\def\etc{{\it etc.}}
\def\eg{{\it e.g.}}
\def\ie{{\it i.e.}}
\def\cf{{\it cf.}}
\def\qv{{\it q.v.}}
\def\qqv{{\it qq.v.}}
\def\st{s.t.\ }
\def\code{\tt}
\def\setsep{:}
\def\concat{\mathbin{|}}

\renewcommand{\thefootnote}{\fnsymbol{footnote}}
\newcommand{\prescite}[1]{\footnote{\cite{#1}}}
\newcommand{\prestext}[1]{\footnotetext{\cite{#1}}}
\newcommand{\emaillink}[1]{\href{mailto:#1}{\nolinkurl{#1}}}

\setlength{\parskip}{0.5em}

% Tables

\newcolumntype{L}[1]{>{\raggedright\arraybackslash\tiny}p{#1}}

% Program listings
\definecolor{listingbaground}{rgb}{.9,.9,.9}

\lstset{
	language=C,
	backgroundcolor=\color{listingbaground},
	extendedchars=true,
	basicstyle=\fontsize{5pt}{6pt}\ttfamily,
	showstringspaces=false,
	showspaces=false,
	numbers=left,
	numberstyle=\fontsize{5pt}{6pt}\ttfamily\color{gray},
	numbersep=5pt,
	tabsize=2,
	breaklines=true,
	showtabs=false,
	captionpos=b,
	xleftmargin=5mm,
	escapeinside={££}{££},
	framexleftmargin=5mm,
	showlines=true
}

% https://tex.stackexchange.com/questions/89574/language-option-supported-in-listings
\lstdefinelanguage{JavaScript}{
	keywords={typeof, new, true, false, catch, function, return, null, catch, switch, var, if, in, while, do, else, case, break, try, const},
	keywordstyle=\color{blue}\bfseries,
	ndkeywords={class, export, boolean, throw, implements, import, this},
	ndkeywordstyle=\color{violet}\bfseries,
	identifierstyle=\color{black},
	sensitive=false,
	comment=[l]{//},
	morecomment=[s]{/*}{*/},
	commentstyle=\color{teal}\ttfamily,
	stringstyle=\color{red}\ttfamily,
	morestring=[b]',
	morestring=[b]"
}

\lstdefinelanguage{QML}{
	keywords={typeof, new, true, false, catch, function, return, null, catch, switch, var, if, in, while, do, else, case, break, id, property, bool, string, int, decimal, target},
	keywordstyle=\color{blue}\bfseries,
	ndkeywords={Page, Example, Timer, Column, PageHeader, TextSwitch, Label},
	ndkeywordstyle=\color{violet}\bfseries,
	identifierstyle=\color{black},
	sensitive=true,
	comment=[l]{//},
	morecomment=[s]{/*}{*/},
	commentstyle=\color{teal}\ttfamily,
	stringstyle=\color{red}\ttfamily,
	morestring=[b]',
	morestring=[b]"
}

\lstdefinelanguage{cpp}{
	keywords={class, public, private, protected, try, throw, catch, this, return, null, switch, if, while, do, else, case, break, for, bool, void, const, static, volatile, int, using, virtual},
	keywordstyle=\color{blue}\bfseries,
	ndkeywords={Q_OBJECT, Q_CLASSINFO, Q_PROPERTY, Q_SLOTS, Q_SIGNALS, signals, slots, READ, WRITE, NOTIFY, QString, QUrl, CefString, CefRefPtr, qreal, nsresult, NS_IMETHODIMP, NS_ENSURE_SUCCESS, nsCOMPtr, nsString, nsCString},
	ndkeywordstyle=\color{violet}\bfseries,
	identifierstyle=\color{black},
	sensitive=false,
	comment=[l]{//},
	morecomment=[s]{/*}{*/},
	commentstyle=\color{teal}\ttfamily,
	stringstyle=\color{red}\ttfamily,
	morestring=[b]',
	morestring=[b]"
}

\lstdefinelanguage{idl}{
	keywords={interface, public, private, void, scriptable},
	keywordstyle=\color{blue}\bfseries,
	ndkeywords={in, out, uuid},
	ndkeywordstyle=\color{violet}\bfseries,
	identifierstyle=\color{black},
	sensitive=false,
	comment=[l]{//},
	morecomment=[s]{/*}{*/},
	commentstyle=\color{teal}\ttfamily,
	stringstyle=\color{red}\ttfamily,
	morestring=[b]',
	morestring=[b]"
}

\lstdefinelanguage{ipdl}{
	keywords={nested, sync, async, protocol, child, parent},
	keywordstyle=\color{blue}\bfseries,
	ndkeywords={nsString, nsCString},
	ndkeywordstyle=\color{violet}\bfseries,
	identifierstyle=\color{black},
	sensitive=false,
	comment=[l]{//},
	morecomment=[s]{/*}{*/},
	commentstyle=\color{teal}\ttfamily,
	stringstyle=\color{red}\ttfamily,
	morestring=[b]',
	morestring=[b]"
}

\lstdefinelanguage{sh2}{
	keywords={dbus, send, zypper, python3},
	keywordstyle=\color{blue}\bfseries,
	ndkeywords={session, type, print, reply, dest},
	ndkeywordstyle=\color{violet}\bfseries,
	identifierstyle=\color{black},
	sensitive=false,
	comment=[l]{\#},
	morecomment=[s]{/*}{*/},
	commentstyle=\color{teal}\ttfamily,
	stringstyle=\color{red}\ttfamily,
	morestring=[b]',
	morestring=[b]"
}

% To allow the theme files to be placed in a subdirectory
\makeatletter
	\def\beamer@calltheme#1#2#3{%
		\def\beamer@themelist{#2}
		\@for\beamer@themename:=\beamer@themelist\do
		{\usepackage[{#1}]{\beamer@themelocation/#3\beamer@themename}}}

	\def\usefolder#1{
		\def\beamer@themelocation{#1}
	}
	\def\beamer@themelocation{}

% Quotation styling
% conditional for xetex or luatex
\newif\ifxetexorluatex
\ifxetex
  \xetexorluatextrue
\else
  \ifluatex
    \xetexorluatextrue
  \else
    \xetexorluatexfalse
  \fi
\fi
%
\ifxetexorluatex%
  \usepackage{fontspec}
  \newfontfamily\quotefont[Ligatures=TeX]{Source Sans Pro} % selects Libertine as the quote font
\else
  \usepackage[utf8]{inputenc}
  \usepackage[T1]{fontenc}
  \newcommand*\quotefont{\fontfamily{Source Sans Pro}} % selects Libertine as the quote font
\fi

\newcommand*\quotesize{60} % if quote size changes, need a way to make shifts relative
% Make commands for the quotes
\newcommand*{\openquote}
   {\tikz[remember picture,overlay,xshift=-3ex,yshift=-2.5ex]
   \node (OQ) {\quotefont\fontsize{\quotesize}{\quotesize}\selectfont``};\kern0pt}

\newcommand*{\closequote}[1]
  {\tikz[remember picture,overlay,xshift=3ex,yshift={#1}]
   \node (CQ) {\quotefont\fontsize{\quotesize}{\quotesize}\selectfont''};}

% select a colour for the shading
\definecolor{shadecolor}{HTML}{ffffff}

\newcommand*\shadedauthorformat{\emph} % define format for the author argument

% Now a command to allow left, right and centre alignment of the author
\newcommand*\authoralign[1]{%
  \if#1l
    \def\authorfill{}\def\quotefill{\hfill}
  \else
    \if#1r
      \def\authorfill{\hfill}\def\quotefill{}
    \else
      \if#1c
        \gdef\authorfill{\hfill}\def\quotefill{\hfill}
      \else\typeout{Invalid option}
      \fi
    \fi
  \fi}
% wrap everything in its own environment which takes one argument (author) and one optional argument
% specifying the alignment [l, r or c]
%
\newenvironment{shadequote}[2][l]%
{\authoralign{#1}
\ifblank{#2}
   {\def\shadequoteauthor{}\def\yshift{-2ex}\def\quotefill{\hfill}}
   {\def\shadequoteauthor{\par\authorfill\shadedauthorformat{#2}}\def\yshift{2ex}}
\begin{snugshade}\begin{quote}\openquote}
{\shadequoteauthor\quotefill\closequote{\yshift}\end{quote}\end{snugshade}}


